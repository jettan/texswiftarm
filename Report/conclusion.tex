\chapter{Conclusion}
The constant struggle against the constraints of the TV was one of the things that made the project challenging and enjoyable. This is one of the reasons we started the project in the third quarter, working part-time to set up a TV ready to use in the fourth quarter. Developing on a device that nobody has really developed for before and having to solve new and unexpected problems along the way made this a very educational project.
In the end, we succeeded in implementing a peer-to-peer application that runs natively on the TV, unlike other solutions using for example a set-top box. \footnote{\url{http://durao.net/2010/06/01/oplay-hdp-r1-\%E2\%80\%93-bittorrent-client-with-firmware-1-27/}}
With this, we are the first in the world to have implemented a fourth generation peer-to-peer application that runs natively on a TV.

Based on the final product it is safe to say that peer-to-peer applications involving streaming, downloading and uploading of files are possible on smart TV\textquotesingle s. Due to hardware limitations the application was never able to reach its full potential but the results are very promising for future televisions with better hardware. Despite the hardware issues, the technology does work which makes our application the first fourth generation peer-to-peer file sharing application on a smart tv in the world.

Another important aspect, next to better hardware, as to whether more advanced applications will be developed for televisions is the amount of freedom manufacturers are willing to give to developers. As it stands, television manufacturers seem very wary of giving any access to their TV\textquotesingle s, which is very understandable from a security point of view. However, if the full potential of the smart TV is to be reached, developers must be given full access to develop advanced programs or else the smart TV will remain a gimmick with limited functionality.
