\chapter{Problem Statement}
\label{sec:problems}
The goal of this project is to create an application that runs on a Samsung SmartTV and uses the libswift peer-to-peer engine to download, upload and stream files. To create an application for a Samsung SmartTV\footnote{\url{http://www.samsung.com/nl/experience/tv/smarttv/}} a software development kit has been provided which allows programmers to create apps using JavaScript, HTML, CSS and Flash.\\\\
A Samsung SmartTV however runs a linux kernel in which we can properly run our download-engine which is written in C++. To be able to do this we must gain root access to the TV, so it becomes possible to operate in the linux environment (See also \hyperref[sec:orientation]{orientation report}). A problem that arises here is that the SDK front-end will be completely separated from the back-end where libswift will run. A solution has to be found to escape from this sandboxed environment. It is also needed to provide a communication mechanism between TV\textquotesingle s,
so that content can be found and shared between TV\textquotesingle s.

Problems to be solved
\begin{itemize}
\item Root a Samsung television
\item Create a C++ back-end with swift
\item Break out of the sandbox
\item Create a JavaScript front-end
\item Inter-TV communication
\end{itemize}

This application will be a proof of concept to show that televisions are capable of running peer-to-peer software natively, without any attached
devices, in this case with the help of the libswift download-engine. Therefore usability is not the primary goal of this application, 
although the application will be designed to be as usable as possible.
