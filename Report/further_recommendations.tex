\chapter{Recommendations}
\label{sec:recommendations}

A major limitation during this project was the hardware in the television. The application we built depends on reasonably heavy python programs and requires quite a lot of memory when streaming. Although the CPU seemed to be able to keep up, the lack of memory was very noticable when streaming. Thankfully, as technology improves and becomes cheaper, televisions will get better hardware and it will not be long untill they will be able to easily run application like this one. For future versions of peer-to-peer applications on televisions, better hardware is a must.

Once televisions are powerfull enough to handle high quality video streaming, a better version of the application can be created. Although we were very happy with the back-end of our application, the GUI part was not as refined because our focus was on the back-end. In order to make the application more user friendly the GUI needs to be improved. Thanks to the way our application is made, any GUI that can send HTTP requests and parse XML is able to interface with our application which makes changing to a completely different GUI very easy if necessary.

At the moment Dispersy and DHT are built into Tribler. When we needed both these services we were forced to run Tribler with a bunch of modules switched off, just leaving Dispersy, DHT and some other necessary elements. It would be better if Dispersy and DHT could be run as independent modules.
