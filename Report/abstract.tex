\chapter*{Summary}
Since their invention, TV\textquotesingle s have become one of the most popular media devices and can be found in almost every livingroom in the 
world. For a long time, the functionality of the TV stayed the same: the ability to view television programs at certain fixed times of the day.
Recently there has been development in the television market adding computing power and internet connectability to televisions.
These new features open a whole new world of possibilities.

The goal of this project was to create an application that runs on a Samsung SmartTV and uses the libswift peer-to-peer engine to download, upload 
and stream files. To create an application for a Samsung SmartTV a software development kit has been provided which allows programmers to create 
apps using JavaScript, HTML, CSS and Flash. This software development kit was used to create the front-end of our application. 
The front-end consists of an internal media player to handle streaming content and media playback found
 on an external USB device.

A Samsung SmartTV runs a linux kernel in which we can properly run our download-engine which is written in C++.
To be able to do this we gained root access to the TV, so it became possible to operate in the linux environment. After this was done, the back-end for our application was implemented in C++. In order to make the front-end communicate with the back-end, we used the client-server architecture where the front-end acts as a client and the back-end as a server.

It was also needed to provide a communication mechanism between TV\textquotesingle s, so that content could be found and shared between TV\textquotesingle s. This was achieved by using the Dispersy and DHT modules developed by the
tribler team. These modules were, however, implemented in Python. In order to make use of this, a part of our application is implemented in Python.

Several steps were taken in order to develop this application. An analysis of requirements was made, which serves as the foundation of the design
of our application. A design following the client-server pattern was made, after which class and sequence diagrams were created.

During development we encountered a lot of problems, mostly because the TV runs a stripped version of the linux kernel. The consequence was that a
lot of things were not available on the TV, so we had to cross-compile the necessities for the TV or use a binary compatible development platform. 
Other problems we encountered were for example that the TV sends TCP RST packages to itself. Also, problems during implementation occured when we 
used threads for the implementation of the HTTP server.

Even though we encountered these problems, we were still able to maintain our schedule and develop everything in time.
By doing several things together and by dividing the work correctly we were able to overcome the mentioned problems.
 The result is the first fourth generation peer-to-peer application in the world running natively on a Samsung SmartTV.
 Even though this is an achievement we are proud of, the application is still a proof of concept since current TV\textquotesingle s are not strong enough to run the application smoothly.
 This might change in the future, when TV\textquotesingle s become more powerful.
