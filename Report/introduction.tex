\chapter{Introduction}
Since their invention, TV\textquotesingle s have become one of the most popular media devices and can be found in almost every livingroom in the world. For a long time, the functionality of the TV stayed the same: the ability to view television programs at certain fixed times of the day. Recently there has been development in the television market adding computing power and internet connectability to televisions. These new features open a whole new world of possibilities. One of these possibilities, and the one our project focusses on, being the ability to stream video content on demand over the Internet.
\\\\
The reason peer-to-peer streaming is such an attractive feature is because it is at the base of video on demand technology. It has always been possible to view video content on televisions but the programs are always at set times and people are not always available at certain times to watch the programs they would like to see. Streaming content on demand would solve this problem completely by allowing users to view the content they are interested in whenever they like. Current technology is capable of providing this service now but it has not yet become popular in televisions. Our goal is to create an application that fulfills this need and set a step in the direction of on demand peer-to-peer video content.
\\\\
The client, but also supervisor of this project is Dr.Ir. Johan Pouwelse from the Parallel and Distributed Systems group on the faculty of EEMCS of the Delft University of Technology.
He is the scientific director of several peer-to-peer research initiatives and also the founder and leader of the Tribler\cite{tribler} \cite{tribler2} peer-to-peer research team.
Tribler is an open source peer-to-peer client that is completely decentralised.
The application to be implemented is required to have the same download-engine, libswift,\cite{swift} as Tribler excluding the part that handles torrents.
The application comes along with an internal media player to handle streaming content and possibly media playback found on an external USB device and should run natively on the TV.
\\\\
Libswift is a lightweight 4th generation peer-to-peer based transport protocol for content dissemination
that can run on top of other protocols, such as UDP, TCP, HTTP, or as RTP profile.
It can be used for both live streaming and conventional downloading purposes.
4th generation peer-to-peer software means that it is fully self-organised, removing the need for any server.\cite{4g}
\\\\
Because libswift is a very generic protocol, it is easy to merge with the existing technology of the TV.
Since the smart TV has limited memory and CPU resources,
it is necessary that the application is built as lightweight as possible.
The streaming capabilities also enables the user to maximise utility of the TV,
while still being able to download data and store it.
A major drawback is however that downloading and streaming content is limited to the memory of the TV.
Because of this limitation, we were only able to create a proof of concept application, rather than an application which can be used immediately
by end users.
