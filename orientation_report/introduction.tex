\chapter{Introduction}

In the third year of the Computer Science course in Delft, students have to do their Bachelor project, 
where they have to work under supervision of a professor.

This report is an orientation report for our Bachelor Project 2011-2012.
The goal of this project is to develop a 4th generation peer-to-peer application to be used on a Samsung Smart TV.
In this report we will set out the possible solutions that were explored for this goal,
the final solution chosen and progress made to begin development of the application itself.
Additionally, some technical background information and a planning for the rest of the project are included.

The client, but also supervisor of this project is Dr.Ir. Johan Pouwelse from the Parallel and Distributed Systems group on the faculty of EEMCS of the Delft University of Technology.
He is the scientific director of several peer-to-peer research initiatives and also the founder and leader of the Tribler peer-to-peer research team.
Tribler is an open source peer-to-peer client that is completely decentralised.
The application to be implemented is required to have the same download-engine, libswift, as Tribler 
along with an internal media player to handle streaming content and possibly media playback found on an external USB device.

Libswift is a lightweight 4th generation peer-to-peer based transport protocol for content dissemination
that can run on top of other protocols, such as UDP, TCP, HTTP, or as RTP profile.
It can be used for both live streaming and conventional downloading purposes.
4th generation peer-to-peer software means that it is fully self-organised, removing the need for any server\footnote{\url{http://www.tribler.org/trac/wiki/4thGenerationP2P}}.

Because libswift is a very generic protocol, it should be easy to merge with the existing technology of the TV.
Since the Smart TV has limited memory and CPU resources, 
it is necessary that the application to be build is as lightweight as possible.
The streaming capabilities also enables the user to maximise utility of the TV,
while still able to download data and store it.
A major drawback is however that downloading and streaming content is limited to the memory of the TV.
