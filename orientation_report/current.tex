\chapter*{Current Progress}
\label{chap:current}

In this section, we will describe what we have been workin on during the third quarter as preparement for the TV development.
Our first choice was to implement the application to run on the TV natively.
Not only because it's challenging, but also because we believe it is possible for the TV to run it natively.
The idea of having a TV without a set-top box or another device running the to be implemented application attracted us to do so.
Moreover, since the Tribler team already ported libswift for Android and made a demo application,
it wouldn't be as fun as implementing the application to run native on the TV.
Should the TV be unrootable, we would have chosen this solution.
Implementing the application to run on the Raspberry Pi was our last backup solution.
Not only due to the unknown availibility of the device,
but also because developing an application for it wouldn't be much different than developing an application for your own computer.

\section*{Gaining root access to the TV}
Initially we believed getting root access of the TV would be a simple step,
but it turned out to be harder than we thought.
To root the TV we planned on using the method provided by SamyGo (See \hyperref[sec:smarttv]{how to root\ref*{sec:smarttv}},
but because the firmware version of our TV was the latest,
we could not use the SamyGo tool\footnote{The SamyGo tool requires the TV to be at firmware <= 1019.0, whereas our TV was at 1021.0.}.

Since there was no official way to downgrade to a lower firmware version,
we decided to try to trick the TV into downgrading.
The idea was to create our own firmware which would be an old version of the firmware repackaged with a higher version number.
Since Samsung provided access to the source of all the firmware versions on the Internet,
all that needed to be done was to change the version number and repackage it.
Sadly Samsung incorporates a salted signature into the packaging of the firmware.
While the SamyGo team possesses a tool to decrypt the packages succesfully,
there was no way to repackage them back, so this plan ended up failing.

During our own attempts to downgrade the firmware we had contacted the developers at SamyGo to ask 
whether they had a solution for our firmware version or an approximate date for a new root kit.
After some conversation they agreed to give us access to a server they used to spoof the Samsung online firmware updating server.
Using the same technique we attempted offline, the idea was to trick the TV into downgrading by making it think it is upgrading.
The online version however, succeeded to downgrade the TV to firmware 1015.0.
Apparently a signature was not needed or was circumvented in some other way.
In return of this favour, we promised we would put some time and effort into a method for gaining root access to the newest firmware version.
The end result, however, was a TV with root access with which we could continue to the next step, getting libswift to run.
After rooting the TV, it was possible to transfer files from and to the TV using FTP and to log into the TV using a netcat shell.

\section*{Compiling libswift for the TV}
Even though we have compiled libswift for ARM machines before, this still proved to be a non-trivial task.
Because the root file system on the TV is read-only, adding libraries like libevent, on which libswift depends was out of the question.
This caused problems when trying to dynamically link libraries.
By setting the 'static' flag on, libraries can be linked statically and solving this problem.

The problem that costed most of our time, however, 
had to do with the transfer of the compiled libswift to the TV.
The libswift library would be compiled on a laptop and then transferred to the TV using FileZilla, an FTP client.
After a long time of wondering why the compiled library would not run or even recognized as executable file by gdb on the TV,
we discovered that FileZilla automatically decides whether the file to be transferred is an ASCII file or a binary file.
For some reason, FileZilla treated our compiled library as an ASCII file.
Once we switched to using FTP in the command line and setting the transfer mode to binary this problem was resolved and got libswift running.

\section*{Running libswift on the TV}
Once libswift was compiled and running on the TV we were finally able to actually test whether the TV could handle it.
For this we tried a number of video and music files of different quality streamed from a laptop to the TV and from the TV to the laptop.
Videos up to a 720p with 24 frames per second went seamlessly.
The TV's media player limitations, however, still apply.\footnote{http://www.samsung.com/uk/consumer/learningresources/media2.0/usb\_faq.html}
When feeded 1080p video files, the TV begun to have some troubles.
We made a demonstration video of these results which can be seen on YouTube\footnote{http://www.youtube.com/watch?v=MUCUMA0LVgc},
this is the same video mentioned earlier.

