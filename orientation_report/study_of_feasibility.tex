\chapter{Study of Feasibility}

In this section, we will evaluate all solutions to study the feasibility of the whole project.
Afterwards, we will select one solution to carry out and use the other solutions as backup plans.

The preparations for the first solution should not be hard to carry out, 
especially since all the steps needed to be taken are well documented on the Internet.
Also, we can get help from the SamyGo developers when needed\footnote{\url{http://forum.samygo.tv}}.

The second solution is in essence the same as the first solution, but on another platform than a TV (such as the Raspberry Pi). 

As for the third solution, 
our client already possesses knowledge regarding libswift applications for Android.
Furthermore, developing applications for Android is not hard 
because its extensive API and the many possibilities it offers\footnote{\url{http://developer.android.com/reference/packages.html}}.
So the third solution is also feasible.

The diversity of possibilities makes the project very feasible, 
because if the first solution really does not work out we can use the second solution as backup plan. 
Because the software written for the TV will be the same as the software written for a board like Raspberry PI, 
it is easy to switch between the two solutions. 
The only thing that needs to be done is to cross-compile the software for the correct platform. 
The third solution also makes use of the same Samsung TV model as the previous solutions, 
so switching to the third solution is easy. 

Since the first solution is already quite feasible itself, 
it is best to try that and to use the other solutions as backup plans.
