\chapter*{Target Roadmap}

In order to keep the development of the application structured we have divided the functionality 
we would like to achieve into separate phases. 
Each phase implements a relatively small and overseeable chunk of functionality 
which builds upon and requires the previous phase to be completed. 
Each phase also guarantees a complete and fully functional piece of software. 
For example, this means that if by the end of the project we are unable to implement phase 4 we will 
still have a fully functional application, although with less functionality, as we have finished phase 3.

The phases as described now are aimed primarily at solution 1 although they can be easily modified to 
apply to the other solutions.

The functionality described in the phases could be categorised as must have, 
should have, would have and could have functionality. What exactly each phase will entail is explained below.

\section*{Phase 1}
This phase concentrates on getting the very basics up and running. 
Our application will be built around the libswift library and therefore 
libswift must be compiled and runnable on the TV. 
As libswift is for file transfer over the Internet we will also need to make the 
TV internet connectible and be able to download files. 
To make accessing the TV easier we want to be able to use Telnet or SSH to connect to it.

To summarise:

The TV is Telnet connectible
The TV is Internet connectible
Libswift runs on the TV
You can download files from the Internet via the command line
At the moment, this phase is already finished as we already succeeded in gaining root access and 
have libswift up and running on the TV. See \hyperref[chap:current]{here\ref*{chap:current}} for more about the current progress.
However, although we already have root access on the TV, 
we would like to find a method to gain root access on the newest firmware currently 
available so that our application will be compatible with the latest software. 

\section*{Phase 2}
The next step is to build our own application using functionality from the libswift library. 
This step will also be run from the command line. 
We want to be able to download files using the libswift library and 
be able to pass them on to a media decoder and play them on the screen. 
Another feature that libswift offers is streaming of video rather than 
downloading a file completely and playing it once the download is complete. 
We would like to implement that in this phase as well.

\begin{itemize}
\item Downloaded files are passed on to a media decoder;
\item Video plays on the screen;
\item Download and play functionality;
\item Full streaming functionality.
\end{itemize}

\section*{Phase 3}
This phase is aimed at making a more user friendly application rather than 
the command line interface that was being used up until this point. 
The advantage is that all the download functionality has been implemented in the 
previous phases so all that needs to be done is build a GUI around the the previous phases. 
We would like the application to be controlled using the TV remote control and 
have some hard coded test swarms from which you can download files for demonstration purposes.

\begin{itemize}
\item A GUI controlled by the TV remote control;
\item A file browser;
\item A list of test swarms (hard coded) from which you can download and play;
\end{itemize}

\section*{Phase 4}
In phase 4 we have a number of options. 
Due to the fact that it is probably unrealistic to implement them all within the given time we have decided to split 
them up into three groups. 
Which one we will choose will be decided at a later time during the project based 
on the amount of time we have left and our personal preference.

\subsection*{Phase 4A}
The first option is to get Python, SQLite and M2Crypto working on the TV. 
This would enable us to run `Dispersie', made by the Tribler team and let us implement searching, 
browsing and sharing functionality.

\begin{itemize}
\item Get Python, SQLite and M2Crypto to run;
\item Use this to implement searching, browsing, streaming and sharing functionality.
\end{itemize}

\subsection*{Phase 4B}
Another option is to expand the application by adding a more advanced GUI with search functionality. 
Also TV to TV download and upload would be interesting rather than downloading from a central server. 
We could also add more demo applications and functions.

\begin{itemize}
\item Expanding the GUI;
\item Add Search functionality;
\item TV to TV uploading and downloading;
\item More demo apps;
\end{itemize}

\subsection*{Phase 4C}
The last option is to connect a web cam to the TV and support streaming to multiple other 
TV\textquotesingle s from the TV with the web cam.

\begin{itemize}
\item Live streaming from a web cam to multiple TV\textquotesingle s using the libswift library.
\end{itemize}

\subsection*{Planning}

\begin{table}[h]
	\begin{tabular}{|l|l|}

\hline
	Week & Goals \\ \hline
	23th -- 29th April & Finish Orientation report. \\
	& Attempt to gain root access to newest firmware version of TV. \\
	& Create design for the back-end of our application. \\
	& Create design for the front-end of our application 
	\\ & and find out how to integrate it with the back-end. \\ \hline
	
	30th April -- 6th May & Finish design of application. \\ 
	& Create basic application with calls to libswift. \\ \hline
	
	7th -- 13th May & Start implementation of our application and test suite. \\ \hline
	14th -- 20th May & Continue implementation. \\ \hline
	21st -- 27th May & Continue implementation. \\ \hline 
	28th May -- 3rd June & Start working towards finished product. \\ \hline 
	4th --10th June & Start writing report. \\
	& Prepare code for SIG. \\ \hline
	
	11th -- 17th June & Write report. \\
	& Improve code based on SIG feedback. \\ \hline
	
	18th -- 24th June & Prepare code for SIG. \\ \hline

	\end{tabular}
	\caption{General planning for the whole project}
	\label{tab:planning}
\end{table}
