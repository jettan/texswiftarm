\chapter*{Solutions}

In this section, we will list 3 solutions to develop a peer-to-peer application for the Samsung Smart TV.

\section*{Develop a native application}
\label{sec:smarttv}
The first solution is to implement the peer-to-peer application so that it can be run natively on the TV itself.
The TVs to be used are the UE40D7000\footnote{http://www.samsung.com/nl/consumer/tv-audio-video/televisions/led-tv/UE40D7000LSXXN-spec} 
and the UE46D8000\footnote{http://www.samsung.com/nl/consumer/tv-audio-video/televisions/led-tv/UE46D8000YSXXH-spec}.
In order to run the application native on the TV, root access is required, 
as the TV does not allow external libraries, in particular libswift, to be used in the Samsung SDK.
In fact, one can only make application with HTML, DOM, CSS or JavaScript with Flash support and the built-in media player\footnote{http://www.samsungdforum.com/Devtools/Spec}.

To gain root access, one must follow the steps on \url{http://wiki.samygo.tv/index.php5/Rooting_D_series_arm_cpu_models}.

These are the steps to gain root access:

Create the developer account.
\begin{enumerate}
\item Enter recovery menu on the TV (press info, menu, mute, power while TV is turned off)
\item Set the RS232 on debug and the watchdog off.
\item Save settings and reboot the TV normally.
\item Enter SmartHub.
\item Go to Settings.
\item Create a developer account (name = develop, password any)
\item Exit SmartHub and reboot the TV.
\end{enumerate}

\newpage
Now install the hack:

\begin{enumerate}
\item Enter SmartHub.
\item Login on your developer account.
\item Go to Settings -> Development -> Setting Server IP
\item Enter 46.4.199.222 as IP.
\item Press User Application Synchronization (This could take a long time).
\item Exit Developer menu and SmartHub.
\item Reboot TV.
\item Enter SmartHub.
\item Execute the SamyGo widget.
\end{enumerate} 


With this the TV will be rooted. To cross-compile code for the Samsung TV, 
a cross-compilation toolchain is needed, which can be found at \url{https://opensource.samsung.com}.
With this, we can cross-compile libswift for the TV. Further development will then involve the use of the Samsung SDK with another daemon that handles the libswift calls.

\section*{Developing an application for the Raspberry Pi}
The second solution is to implement an application for an embedded system sporting an arm processor such as the Raspberry Pi\footnote{http://www.raspberrypi.org}
and using the Raspberry Pi as a set-top box on the Samsung Smart TV.
Since the Raspberry Pi already runs a Linux distribution such as Fedora, development can be done on the device itself.
Lots of libraries are already present and we can even make use of strong media players such as mplayer.
In short, the application to be developed would then be a Tribler clone, but only for arm Linux.

\section*{Develop an application for the Galaxy Nexus}
The main idea of this solution is to develop an application that runs on the Samsung Galaxy Nexus with Android Ice Cream Sandwich.
Display will be forwarded to the TV screen by using an MHL cable\footnote{http://www.handtec.co.uk/product.php/6075/samsung-galaxy-nexus-official-mhl-microusb-to-hdmi-adapter}.
Since the Tribler team already ported libswift for the Android successfully,
we can proceed with the development of the application itself when choosing this solution.
