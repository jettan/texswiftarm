\chapter{Usage of Libswift}

Libswift comes as a command line application, the version we use can be 
found at 
\url{https://github.com/jettan/swiftarm/tree/master/swift}. With 
this application, it is possible to upload and download files. Streaming 
audio and video also belongs to the possibilities. These streams can be 
bound to a port which can be used on a media player to play back the 
content. Control of libswift can also be done via TCP.
Unlike conventional peer-to-peer systems, libswift does not require a centralised tracker to operate.
This allows libswift to work completely decentralised, which makes it very robust.
A short video where the command line usage of libswift is 
demonstrated can be found on YouTube\footnote{\url{http://www.youtube.com/watch?v=MUCUMA0LVgc}}. For more 
information on how libswift works, we refer the reader to 
\url{https://datatracker.ietf.org/doc/draft-ietf-ppsp-peer-protocol/}.

For our own application, we will have to use the same functionality as 
the command line application of libswift, but then stripped down to fit 
our needs. In short, we will use the swift.cpp file as an example to 
develop the basics for our application.
